\documentclass[12pt, a4paper, twoside]{report}

% --- Basic Setup ---
% \usepackage[utf8]{inputenc} % Input encoding (UTF-8 is standard for modern LaTeX).
% Note: 'inputenc' is often not explicitly needed with modern LaTeX distributions (TeX Live 2018+),
% as UTF-8 is the default. Uncomment if you encounter encoding issues.
\usepackage[T1]{fontenc}    % Font encoding for correct hyphenation and character display
\usepackage{lmodern}        % Use Latin Modern fonts, a good default for academic documents
\usepackage[german, english]{babel} % Language support: German first for decimal comma, then English
\usepackage{csquotes}       % Context-sensitive quotation marks, required by biblatex for some styles
\usepackage{blindtext}      % For placeholder text. REMOVE THIS PACKAGE FOR YOUR ACTUAL DISSERTATION!

% --- Page Layout & Spacing ---
\usepackage[
    a4paper,
    top=3cm,
    bottom=3cm,
    left=3.5cm,
    right=2.5cm,
    headheight=14.5pt, % Required by fancyhdr for proper header placement
    % showframe % Uncomment this line to visualize the page margins for debugging purposes
]{geometry} % Package for custom page dimensions and margins
\usepackage{setspace}       % For controlling line spacing throughout the document
\doublespacing              % Set default to double spacing, common for dissertation drafts
% You can change this to \onehalfspacing or \singlespacing as needed.

% --- Mathematics & Units ---
\usepackage{amsmath, amssymb, amsfonts} % AMS mathematical packages for advanced math typesetting
\usepackage{bm}             % Bold math symbols (e.g., for vectors \bm{v})
\usepackage{siunitx}        % For correct typesetting of units and numbers (e.g., \SI{10,5}{\kilo\gram})
% Configure siunitx to use comma as decimal separator, aligning with your preferences
\sisetup{output-decimal-marker={,}}

% --- Graphics & Tables ---
\usepackage{graphicx}       % For including images (e.g., .png, .jpg, .pdf)
\usepackage{float}          % Provides the [H] float specifier for precise placement of figures/tables
\usepackage{caption}        % Customizing captions for figures and tables
\usepackage{subcaption}     % For creating subfigures and subtables within a single float
\usepackage{booktabs}       % For professional-looking tables with better horizontal rules
\usepackage{multirow}       % For cells spanning multiple rows in tables
\usepackage{longtable}      % For tables that can span multiple pages

% --- Hyperlinks & PDF Metadata ---
\usepackage[
    colorlinks=true,        % Make links colored instead of boxed outlines
    linkcolor=blue,         % Color for internal links (e.g., TOC, section references)
    citecolor=blue,         % Color for citation links
    urlcolor=blue,          % Color for URL links
    filecolor=magenta,      % Color for file links
    pdftitle={\dissertationtitle}, % Sets the PDF document title metadata
    pdfauthor={\dissertationauthor}, % Sets the PDF document author metadata
    pdfsubject={Dissertation},  % Sets the PDF document subject metadata
    pdfkeywords={\dissertationkeywords}, % Sets PDF document keywords
    pdfproducer={LaTeX with pdfLaTeX}, % Indicates the software used to produce the PDF
    pdfcreator={LaTeX},         % Indicates the primary creator
]{hyperref} % Creates clickable hyperlinks within the PDF and embeds metadata

% --- Headers and Footers ---
\usepackage{fancyhdr}       % Package for customizing headers and footers
\pagestyle{fancy}           % Apply the 'fancy' style to all pages
\fancyhf{} % Clear all header and footer fields initially
% Page numbers: placed on the right on odd-numbered pages (RO) and left on even-numbered pages (LE)
\fancyhead[RO,LE]{\thepage}
% Chapter title on the left on odd-numbered pages (LO), converted to uppercase, then back to normal case
\fancyhead[LO]{\nouppercase{\rightmark}}
% Section title on the right on even-numbered pages (RE), converted to uppercase, then back to normal case
\fancyhead[RE]{\nouppercase{\leftmark}}
\renewcommand{\headrulewidth}{0.4pt} % Sets the thickness of the line under the header
\renewcommand{\footrulewidth}{0pt}   % Sets the thickness of the line above the footer (0pt means no line)

% --- Bibliography ---
% Using biblatex with biber backend for modern, flexible bibliography management
\addbibresource{references.bib} % Specifies your bibliography file (kept in root for this example)
\usepackage[
    backend=biber,      % Use Biber for processing bibliography (more powerful than BibTeX)
    style=authoryear,   % Primary citation style (e.g., (Smith, 2023))
    citestyle=authoryear, % Citation command style (can be different from bibliography style)
    sorting=nty,        % Sort bibliography by name, title, year
    natbib=true,        % Allows using natbib commands like \citet, \citep alongside biblatex commands
    doi=false,          % Do not print DOI in the bibliography (set to true if you want them)
    url=false,          % Do not print URLs
    isbn=false,         % Do not print ISBN
    eprint=false,       % Do not print eprint
]{biblatex}

% --- Glossary and Abbreviations ---
% Package for creating lists of abbreviations, symbols, and a general glossary
\usepackage[
    nogroupskip, % Prevents extra vertical space between glossary groups
    acronyms,    % Enable the list of acronyms
    symbols,     % Enable the list of symbols (uncomment if you need it)
    toc          % Add glossary/abbreviations to the Table of Contents
]{glossaries-extra}
\setglossarystyle{list} % Choose a style for your glossary/abbreviations list
\makeglossaries % Command to initialize and create necessary files for glossaries

% --- Input Dissertation Metadata ---
% All custom dissertation information is now defined in a separate file.
% frontmatter/dissertation_metadata.tex
% Define your dissertation-specific details here.
% These commands make it easy to update your information consistently throughout the document
% and on the title page.

\newcommand{\dissertationtitle}{The Grand Unified Theory of Everything: A Modern Perspective}
\newcommand{\dissertationauthor}{Dr. Ima Student}
\newcommand{\dissertationuniversity}{University of Leipzig}
\newcommand{\dissertationfaculty}{Faculty of Physics and Earth Sciences}
\newcommand{\dissertationdepartment}{Institute for Theoretical Physics}
\newcommand{\dissertationdegree}{Doctor of Natural Sciences (Dr. rer. nat.)}
\newcommand{\dissertationsubmissiondate}{July 30, 2025}
\newcommand{\dissertationsupervisorone}{Prof. Dr. A. Einstein}
\newcommand{\dissertationsupervisortwo}{Prof. Dr. M. Curie}
\newcommand{\dissertationkeywords}{Unified Field Theory, Quantum Gravity, Standard Model, Theoretical Physics, LaTeX, Dissertation}


% --- Custom Title Page Definition ---
% This command defines the entire layout of your dissertation's title page.
% It uses variables defined in dissertation_metadata.tex
\newcommand{\makemydissertationtitlepage}{
    \thispagestyle{empty} % Ensure no page number appears on the title page
    \begin{center}
        \vspace*{2cm} % Vertical space from the top of the page

        % Placeholder for your university logo.
        % Ensure 'uni-logo.png' is in the same directory as your .tex file,
        % or provide the full path (e.g., 'images/uni-logo.png').
        {\includegraphics[width=0.3\textwidth]{uni-logo.png}\par}
        \vspace{1cm}

        {\LARGE \bfseries \dissertationuniversity \par} % University name
        {\Large \dissertationfaculty \par}    % Faculty name
        {\large \dissertationdepartment \par} % Department name
        \vspace{2cm}

        {\Huge \bfseries \dissertationtitle \par} % Dissertation title
        \vspace{1.5cm}

        {\Large Dissertation \par} % Type of document
        {\large zur Erlangung des Grades eines \dissertationdegree \par} % German degree text
        {\large (Submitted for the degree of \dissertationdegree) \par} % English degree text
        \vspace{1cm}

        {\large vorgelegt von \par} % Submitted by
        {\Huge \bfseries \dissertationauthor \par} % Author's name (large and bold)
        \vspace{2cm}

        {\large Erstgutachter: \dissertationsupervisorone \par} % First supervisor
        {\large Zweitgutachter: \dissertationsupervisortwo \par} % Second supervisor
        \vspace{1.5cm}

        {\large Eingereicht am: \dissertationsubmissiondate \par} % Submission date
    \end{center}
    \newpage % Start the next content (Abstract) on a new page
}

% --- Custom Abstract Environment ---
% Defines a clean environment for your abstract, typically unnumbered and centered.
\newenvironment{abstractdissertation}{
    \cleardoublepage % Ensure abstract starts on a new, odd-numbered page
    \thispagestyle{empty} % No page number for the abstract page
    \null\vfill % Vertically center the abstract title on the page
    \begin{center}
        \bfseries \Huge Abstract
    \end{center}
    % Add "Abstract" to the Table of Contents without a chapter number
    \addcontentsline{toc}{chapter}{Abstract}
    \vspace{1cm} % Space after the abstract title
}{\vfill\null} % Ends the abstract environment, balancing vertical space

% --- Example List of Abbreviations Definitions ---
% Define your acronyms and their full forms here.
\newacronym{latex}{LaTeX}{A document preparation system}
\newacronym{guten}{GUTEN}{Grand Unified Theory of Everything}
\newacronym{phd}{PhD}{Doctor of Philosophy}
\newacronym{cern}{CERN}{European Organization for Nuclear Research}


% --- Document Start ---
\begin{document}

% --- Front Matter ---
% The \frontmatter command sets page numbering to roman numerals (i, ii, iii...)
% and makes chapters unnumbered (e.g., "Abstract" instead of "Chapter 1 Abstract").
\frontmatter

\makemydissertationtitlepage % Call the custom command to generate the title page

% --- Dedication (Optional) ---
% Uncomment and fill in if you have a dedication
% \cleardoublepage
% \chapter*{Dedication}
% \addcontentsline{toc}{chapter}{Dedication}
% % frontmatter/dedication_content.tex
% Content for your dedication
Dedicated to my parents, whose unwavering support made this journey possible.
 % Content for your dedication

% --- Declaration (Optional) ---
% Uncomment and fill in if you have a declaration
% \cleardoublepage
% \chapter*{Declaration}
% \addcontentsline{toc}{chapter}{Declaration}
% % frontmatter/declaration_content.tex
% Content for your declaration of originality
I hereby declare that this dissertation is my own original work and has not been submitted for any other degree or qualification at any other university. All sources used have been acknowledged.
 % Content for your declaration

% --- Abstract Section ---
\begin{abstractdissertation}
    % frontmatter/abstract_content.tex
% Content for your dissertation abstract
This dissertation explores the theoretical framework for the Grand Unified Theory of Everything (\gls{guten}), attempting to reconcile quantum mechanics with general relativity. We present novel approaches to quantum gravity and discuss their implications for particle physics beyond the Standard Model. The research leverages advanced computational methods and draws upon recent experimental data from facilities like \gls{cern}. This work also serves as a practical example of high-quality document preparation using \gls{latex}.
 % Content for your abstract
    \vspace{1em} % Small vertical space before keywords
    \noindent\textbf{Keywords:} \dissertationkeywords
\end{abstractdissertation}

% --- Acknowledgments Section ---
\cleardoublepage % Ensures acknowledgments start on a new page
\chapter*{Acknowledgments} % Unnumbered chapter title for Acknowledgments
\addcontentsline{toc}{chapter}{Acknowledgments} % Add "Acknowledgments" to TOC
\thispagestyle{plain} % Page number at the bottom center for this page
% frontmatter/acknowledgments_content.tex
% Content for your acknowledgments
I would like to express my sincere gratitude to my supervisors, \dissertationsupervisorone{} and \dissertationsupervisortwo, for their invaluable guidance, support, and insightful discussions throughout this research. Their expertise and encouragement were instrumental in completing this dissertation. I also extend my thanks to the University of Leipzig for providing the resources and environment necessary for this study. Finally, I am grateful to my family and friends for their unwavering support.
\blindtext[1] % More placeholder text
 % Content for your acknowledgments

% --- Table of Contents, List of Figures, List of Tables, List of Abbreviations ---
\cleardoublepage % Ensures TOC starts on a new page
\tableofcontents % Generates the Table of Contents

\cleardoublepage % Ensures List of Figures starts on a new page
\listoffigures % Generates the List of Figures

\cleardoublepage % Ensures List of Tables starts on a new page
\listoftables % Generates the List of Tables

\cleardoublepage % Ensures List of Abbreviations starts on a new page
\printglossary[type=acronym, title=List of Abbreviations] % Prints the List of Abbreviations

% --- Main Matter ---
% The \mainmatter command switches to Arabic page numbering (1, 2, 3...)
% and makes chapters numbered (e.g., "Chapter 1 Introduction").
\mainmatter

% --- Include all chapters ---
% Use \include for main chapters. Each \include will start a new page.
% chapters/00_introduction/introduction.tex
\chapter*{Introduction} % Unnumbered chapter, as per original Quarto layout
\addcontentsline{toc}{chapter}{Introduction} % Add to TOC as unnumbered
\label{chap:introduction}
This dissertation delves into the complex and fascinating realm of unified field theories, aiming to provide a comprehensive overview and propose new theoretical insights. The pursuit of a \gls{guten} has been a central theme in theoretical physics for decades, attempting to describe all fundamental forces of nature within a single, consistent framework \cite{Smith2023}. This introduction will set the stage for our exploration, outlining the historical context, the current state of knowledge, and the specific questions this research seeks to address.
\blindtext[2] % Placeholder content

% chapters/01_literature_review/literature_review.tex
\chapter{Literature Review}
\label{chap:literature_review}
A thorough review of existing literature reveals various approaches to quantum gravity, including string theory, loop quantum gravity, and emergent gravity theories \citep{Smith2023, Bloggs2021}. While each approach offers unique insights, a definitive consensus remains elusive.

% Input sections from separate files
% chapters/01_literature_review/01_background_theory.tex
\section{Background and Motivation}
\label{sec:background}
The Standard Model of particle physics has been incredibly successful in describing three of the four fundamental forces: the strong, weak, and electromagnetic forces. However, gravity remains an outlier, resisting attempts to integrate it seamlessly into the quantum field theory framework. Pioneering work by \citet{Doe2022} and others has laid the groundwork for many of the concepts explored herein.

A cornerstone of modern physics is the mass-energy equivalence, expressed as:
\[ E = mc^2 \]
where \(E\) represents energy, \(m\) denotes mass, and \(c\) is the speed of light in a vacuum. This equation highlights the profound connection between mass and energy. Another fundamental principle is Gauss's law for electricity, which describes the distribution of electric fields:
\[ \nabla \cdot \mathbf{E} = \frac{\rho}{\varepsilon_0} \]
Here, \(\mathbf{E}\) is the electric field, \(\rho\) is the charge density, and \(\varepsilon_0\) is the vacuum permittivity. Our work builds upon these foundational principles.
\blindtext[1]

% chapters/01_literature_review/02_related_work.tex
\section{Related Work and Research Questions}
\label{sec:related-work}
This section reviews specific prior research directly relevant to our current endeavor. Many attempts have been made to unify gravity with quantum mechanics, each with its own set of challenges and successes \citep{Bloggs2021}.

This dissertation aims to answer the following primary research questions:
\begin{enumerate}
    \item How can quantum gravity be formulated in a way that is consistent with both general relativity and quantum mechanics?
    \item What are the observable consequences of a unified theory at accessible energy scales?
    \item Can a complete \gls{guten} provide new insights into dark matter and dark energy?
\end{enumerate}
Our documentation and analysis are meticulously prepared using \gls{latex}, ensuring precision and clarity.
\blindtext[1]


\blindtext[1] % More placeholder content

% Example table (assuming you have a tables/ subfolder here, or img/ for general figures)
\begin{table}[h!]
    \centering
    \caption{Key Parameters from Selected Theoretical Models}
    \label{tab:sample-data}
    \begin{tabular}{l c c S[table-format=2.2]}
        \toprule
        Model & Year Proposed & Number of Dimensions & {Predicted Mass (\si{\mega\electronvolt})} \\
        \midrule
        String Theory (Type I) & 1980s & 10 & 1,25 \\
        Loop Quantum Gravity & 1990s & 4 & 0,87 \\
        Emergent Gravity (Example) & 2010s & 4 & 1,12 \\
        \bottomrule
    \end{tabular}
    \caption*{Note: Predicted masses are illustrative and depend on specific model parameters.}
\end{table}

% Example figure (assuming you have a img/ subfolder in your root, or figs/ for chapter-specific ones)
\begin{figure}[h!]
    \centering
    % You should replace 'conceptual-model.pdf' with your actual image file.
    % Make sure it's located at 'img/conceptual-model.pdf'
    \includegraphics[width=0.7\textwidth]{img/conceptual-model.pdf}
    \caption{A conceptual model illustrating the unification of fundamental forces at high energies.}
    \label{fig:conceptual-model}
\end{figure}
As depicted in Figure \ref{fig:conceptual-model}, the convergence implies a single force governing all interactions at the Planck scale.

% chapters/02_methodology/methodology.tex
\chapter{Methodology}
\label{chap:methodology}
This chapter details the theoretical and computational methodologies employed in this dissertation. We utilize a combination of analytical derivations and numerical simulations to explore the properties of various unified field theories. Our approach involves extending established frameworks by incorporating new symmetry principles and investigating their implications for particle spectra and cosmological phenomena. Specific attention is given to quantum field theory in curved spacetime and the renormalization group flow of gravity.
\blindtext[3] % Placeholder content

% chapters/03_results/results.tex
\chapter{Results}
\label{chap:results}
The results obtained from our theoretical calculations and simulations are presented in this chapter. We demonstrate the emergence of a consistent gravitational interaction from a gauge theory perspective and identify a novel mechanism for mass generation within the unified framework. The predicted particle spectrum is discussed, along with a comparison to existing experimental data from particle accelerators. Furthermore, we present preliminary findings on the cosmological implications of our model, including its potential to explain dark matter properties.
\blindtext[3] % Placeholder content

% chapters/04_discussion/discussion.tex
\chapter{Discussion}
\label{chap:discussion}
In this chapter, we discuss the implications of our results in the broader context of theoretical physics. We critically evaluate the strengths and limitations of our proposed model, contrasting it with other prominent approaches to quantum gravity and grand unification. The consistency of our findings with current experimental observations is assessed, and areas where future experimental verification would be most beneficial are highlighted. We also address potential challenges and future research directions stemming from this work.
\blindtext[3] % Placeholder content

% Add more \include lines for your additional chapters

% --- Back Matter ---
% The \backmatter command typically removes chapter numbering for appendices and bibliography.
\backmatter

\appendix % Marks the start of appendices. Chapters following this command will be numbered A, B, C etc.

% --- Include all appendices ---
% endmatter/appendices/appendix_a.tex
\chapter{Supplementary Data and Derivations}
\label{chap:appendix-a}
This appendix provides supplementary data, detailed mathematical derivations, and additional figures that support the main arguments presented in the dissertation. Here, you will find the full set of field equations, renormalization group equations, and Feynman diagrams used in our calculations. Specific examples of numerical simulation parameters and convergence plots are also included.
\blindtext[2] % Placeholder content

% endmatter/appendices/appendix_b.tex
\chapter{Code Listings}
\label{chap:appendix-b}
This appendix contains relevant code snippets and algorithms developed and used during the course of this research. This includes code for numerical simulations, data analysis, and visualization. All code is provided with appropriate comments and licensing information to facilitate reproducibility and future development.
\blindtext[2] % Placeholder content

% Add more \include lines for your additional appendices

% --- Bibliography ---
\cleardoublepage % Ensures bibliography starts on a new page
\printbibliography[heading=bibintoc, title={Bibliography}] % Prints the bibliography and adds it to the Table of Contents

\end{document}
