% chapters/01_literature_review/literature_review.tex
\chapter{Literature Review}
\label{chap:literature_review}
A thorough review of existing literature reveals various approaches to quantum gravity, including string theory, loop quantum gravity, and emergent gravity theories \citep{Smith2023, Bloggs2021}. While each approach offers unique insights, a definitive consensus remains elusive.

% Input sections from separate files
% chapters/01_literature_review/01_background_theory.tex
\section{Background and Motivation}
\label{sec:background}
The Standard Model of particle physics has been incredibly successful in describing three of the four fundamental forces: the strong, weak, and electromagnetic forces. However, gravity remains an outlier, resisting attempts to integrate it seamlessly into the quantum field theory framework. Pioneering work by \citet{Doe2022} and others has laid the groundwork for many of the concepts explored herein.

A cornerstone of modern physics is the mass-energy equivalence, expressed as:
\[ E = mc^2 \]
where \(E\) represents energy, \(m\) denotes mass, and \(c\) is the speed of light in a vacuum. This equation highlights the profound connection between mass and energy. Another fundamental principle is Gauss's law for electricity, which describes the distribution of electric fields:
\[ \nabla \cdot \mathbf{E} = \frac{\rho}{\varepsilon_0} \]
Here, \(\mathbf{E}\) is the electric field, \(\rho\) is the charge density, and \(\varepsilon_0\) is the vacuum permittivity. Our work builds upon these foundational principles.
\blindtext[1]

% chapters/01_literature_review/02_related_work.tex
\section{Related Work and Research Questions}
\label{sec:related-work}
This section reviews specific prior research directly relevant to our current endeavor. Many attempts have been made to unify gravity with quantum mechanics, each with its own set of challenges and successes \citep{Bloggs2021}.

This dissertation aims to answer the following primary research questions:
\begin{enumerate}
    \item How can quantum gravity be formulated in a way that is consistent with both general relativity and quantum mechanics?
    \item What are the observable consequences of a unified theory at accessible energy scales?
    \item Can a complete \gls{guten} provide new insights into dark matter and dark energy?
\end{enumerate}
Our documentation and analysis are meticulously prepared using \gls{latex}, ensuring precision and clarity.
\blindtext[1]


\blindtext[1] % More placeholder content

% Example table (assuming you have a tables/ subfolder here, or img/ for general figures)
\begin{table}[h!]
    \centering
    \caption{Key Parameters from Selected Theoretical Models}
    \label{tab:sample-data}
    \begin{tabular}{l c c S[table-format=2.2]}
        \toprule
        Model & Year Proposed & Number of Dimensions & {Predicted Mass (\si{\mega\electronvolt})} \\
        \midrule
        String Theory (Type I) & 1980s & 10 & 1,25 \\
        Loop Quantum Gravity & 1990s & 4 & 0,87 \\
        Emergent Gravity (Example) & 2010s & 4 & 1,12 \\
        \bottomrule
    \end{tabular}
    \caption*{Note: Predicted masses are illustrative and depend on specific model parameters.}
\end{table}

% Example figure (assuming you have a img/ subfolder in your root, or figs/ for chapter-specific ones)
\begin{figure}[h!]
    \centering
    % You should replace 'conceptual-model.pdf' with your actual image file.
    % Make sure it's located at 'img/conceptual-model.pdf'
    \includegraphics[width=0.7\textwidth]{img/conceptual-model.pdf}
    \caption{A conceptual model illustrating the unification of fundamental forces at high energies.}
    \label{fig:conceptual-model}
\end{figure}
As depicted in Figure \ref{fig:conceptual-model}, the convergence implies a single force governing all interactions at the Planck scale.
